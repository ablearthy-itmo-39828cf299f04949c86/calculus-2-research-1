\documentclass[a4paper,12pt]{article}

\usepackage{ablpreamble}

\usepackage[inline]{asymptote}
\usepackage{caption}
\usepackage{subcaption}

\newcommand*\diff{\mathop{}\!\mathrm{d}}

\def\asydir{asy}

%\usepackage{svg}

\begin{document}


\begin{asydef}
// Global Asymptote definitions can be put here.
settings.prc=true;
texpreamble("\usepackage{xltxtra,unicode-math}");
texpreamble("\setmainfont{Times New Roman}\setromanfont{Times New Roman}");
texpreamble("\setsansfont{Arial}\setmonofont{Courier New}");
texpreamble("\setmathfont{XITS Math}");
\end{asydef}

%\section{Задача № 1}
%
%\[
%  g(x) =
%  \begin{cases}
%    \sqrt{x - 3}, & x \ge 3, \\
%    \frac{1}{x - 3}, & 0 \le x < 3, \\
%    3 x^2 + 1, & x < 0.
%  \end{cases}
%\]
%
%\begin{enumerate}
%  \item
%    Найдите такую непрерывную функцию \(f(x)\),
%    что \(f'(x) = g(x)\) или докажите, что она не может быть непрерывна.
%  \item
%    Постройте графики функций \(f(x)\) и \(g(x)\) в одном масштабе
%    и расположите их один под другим.
%  \item
%    Проанализируйте графики,
%    сделайте комментарии о виде графика функции \(f(x)\)
%    в зависимости от вида графика функции \(g(x)\).
%\end{enumerate}

\section{Задача № 2}

\subsection{Задача № 2 А}

Вычислите работу, которую нужно затратить на выкачивание воды из резервуара,
представляющего из себя усеченный конус,
у которого радиус верхнего основания равен 1 м, нижнего – 2 м, высота – 3 м.
Удельный вес воды принять \(\rho = \SI{9.81}{\kN\per\cubic\metre}\),
\(\pi = \num{3.14}\)

\subsubsection{Решение}

\begin{figure}[htbp]
  \centering
  \begin{subfigure}[b]{0.47\textwidth}
    \centering
    \asyinclude{asy/2-a-1.asy}
    \caption{Графическая иллюстрация}
  \end{subfigure}
  \hfill
  \begin{subfigure}[b]{0.47\textwidth}
    \centering
    \asyinclude{asy/2-a-2.asy}
    \caption{Схематичный чертёж}
  \end{subfigure}
\end{figure}

\paragraph{Математическая модель}

Обозначим радиус верхнего основания \(r_1 = \mathrm{AH} = 1\),
нижнего --- \(r_2 = \mathrm{OE} = 2\),
а высоту усеченного конуса \(h = \mathrm{OH} = 3\).

Выделим на высоте \(x\) (высоту будем считать от нижнего основания конуса)
слой жидкости толщиной \(\Delta x\).
Чтобы найти его объем,
вначале вычислим \(r = \mathrm{O'D}\) --- радиус сечения на высоте \(x\).
\begin{equation}\label{eq:2a-1}
  r = r_1 + \mathrm{BD}
\end{equation}
Найдём \(BD\) из подобия треугольников
\(\triangle \mathrm{ACE}\) и \(\triangle \mathrm{ABD}\):
\begin{equation}\label{eq:2a-2}
  \mathrm{BD} : \mathrm{CE} = \mathrm{AB} : \mathrm{AC},
  \quad\text{тогда }
  \mathrm{BD} = \frac{\mathrm{CE} \cdot \mathrm{AB}}{\mathrm{AC}}.
\end{equation}

Далее выразим
\(\mathrm{CE} = r_2 - r_1\),
\(\mathrm{AB} = h - x\),
\(\mathrm{AC} = \mathrm{OH} = h\).

При подстановке выражений в формулы \ref{eq:2a-1} и \ref{eq:2a-2} получаем:
\[
  r = r_1 + \frac{\mathrm{CE} \cdot \mathrm{AB}}{\mathrm{AC}}
    = r_1 + \frac{(r_2 - r_1)(h - x)}{h}
\]

Таким образом, при \(\Delta x \to 0\) объем слоя воды толщины \(\Delta x\)
на высоте \(x\) будет равен:
\[\Delta V = \pi r^2 \Delta x.\]

Этот слой нужно поднять на высоту \(h - x\),
значит элементарная работа, затрачиваемая на подъем этого слоя,
будет равна:
\[\diff A = \rho (h - x) \diff V.\]

Тогда работа по выкачиванию всей воды будет равна:
\begin{equation}\label{eq:2a-int}
\begin{split}
  A = \int_{0}^{h} \diff A
    = \int_{0}^{h} \rho (h - x) \diff V
    &= \int_{0}^{h} \rho (h - x) \pi r^2 \diff x \\
    &= \int_{0}^{h} \rho (h - x) \pi
       {\left(r_1 + \frac{(r_2 - r_1)(h - x)}{h}\right)}^2 \diff x
\end{split}
\end{equation}

\paragraph{Вычисления}

Заданные значения подставим в полученную формулу \ref{eq:2a-int}:
\begin{equation*}
\begin{split}
  A &= \pi \cdot 9.81 \int_{0}^{3} (3 - x)
      {\left(1 + \frac{(2 - 1)(3 - x)}{3} \right)}^2 \diff x
    =
      \begin{bmatrix}
        t = 3 - x        & \diff t = - \diff x \\
        t_1 = 3 - 0  = 3 & t_2 = 3 - 3 = 0 \\
      \end{bmatrix} \\
    &= \pi \cdot 9.81 \cdot (-1)
       \int_{3}^{0} t {\left(1 + \frac{t}{3}\right)}^2 \diff t
     = \pi \cdot 9.81
       \int_{0}^{3}
         \left(t + \frac{2 t^2}{3} + \frac{t^3}{9} \right) \diff t \\
    &= \pi \cdot 9.81 \cdot
       \left.
         \left(\frac{t^2}{2} + \frac{2 t^3}{9} + \frac{t^4}{36} \right)
       \right\rvert_{0}^{3}
     = \pi \cdot 9.81 \cdot
     \left(\frac{9}{2} + 6 + \frac{9}{4} \right)
     = \pi \cdot 9.81 \cdot \frac{51}{4}
     \approx 393
\end{split}
\end{equation*}

\textbf{Ответ}: \SI{393}{\kilo\newton\per\metre}.

\subsection{Задача № 2 Б}

Вычислите силу давления воды на пластинку, вертикально погруженную в воду,
считая, что удельный вес воды равен \(\rho = \SI{9.81}{\kN\per\cubic\metre}\).
Результат округлите до целого числа.

\subsubsection{Решение}

\begin{figure}[htbp]
  \centering
  \asyinclude{asy/2-b-1.asy}
  \caption{Графическая иллюстрация}\label{fig:2b}
\end{figure}

\paragraph{Математическая модель}

Обозначим вертикальную диагональ ромба \(h = 4\), а горизонтальную \(w = 2\).
Выделим на глубине \(x\) горизонтальную полосу шириной \(\diff x\).
В силу подобия треугольников
\(\triangle \mathrm{BCD}\) и \(\triangle \mathrm{ACE}\)
(см. рисунок \ref{fig:2b}):
\[
  \frac{\mathrm{BD}}{\mathrm{AE}} = \frac{\mathrm{O_1 C}}{\mathrm{OC}}
  \Leftrightarrow
  \frac{\mathrm{BD}}{w} = \frac{h - x}{h / 2}
  \Leftrightarrow
  \mathrm{BD} = \frac{2w}{h} (h - x)
\]

Однако данная формула справедлива только
если выбранная полоса находится глубже, чем отрезок \(\mathrm{AE}\),
т.е. если \(x \ge \frac{h}{2}\).
Если же \(x < \frac{h}{2}\), то \(\mathrm{BD} = \frac{2w}{h} x\).
Тогда при \(\diff x \to 0\) площадь полосы будет равна
(приближение до площади элементарного прямоугольника):
\begin{equation}
  \begin{cases}
    \diff S = \frac{2w}{h} x \diff x, & x < \frac{h}{2}, \\
    \diff S = \frac{2w}{h} (h - x) \diff x, & x \ge \frac{h}{2}.
  \end{cases}
\end{equation}

Тогда давление на неё будет равно:
\begin{equation}
  \begin{cases}
    \diff P = \rho x \diff S
      = \rho \frac{2w}{h} x^2 \diff x, & x < \frac{h}{2}, \\
    \diff P = \rho x \diff S
      = \rho \frac{2w}{h} (h - x) x \diff x, & x \ge \frac{h}{2}.
  \end{cases}
\end{equation}

Давление на всю пластинку будет равно:
\begin{equation}\label{eq:2b}
\begin{split}
  P &= \int_{0}^{h/2} \diff P + \int_{h/2}^{h} \diff P
    = \int_{0}^{h/2} \rho \frac{2w}{h} x^2 \diff x
     + \int_{h/2}^{h} \rho \frac{2w}{h} (h - x) x \diff x \\
    &= \rho \frac{2w}{h}
     \left(\int_{0}^{h/2} x^2 \diff x + \int_{h/2}^{h} (h - x) x \diff x\right)
\end{split}
\end{equation}

\paragraph{Вычисления}
Подставим в полученную формулу \ref{eq:2b} заданные значения.
\begin{align*}
  P &= 9.81 \cdot \frac{2 \cdot 2}{4}
    \left(\int_{0}^{2} x^2 \diff x + \int_{2}^{4} (4 - x) x \diff x\right) \\
    &= 9.81 \cdot \left(
      \left.\frac{x^3}{3}\right\rvert_{0}^{2}
      + \left.\left(2 x^2 - \frac{x^3}{3}\right)\right\rvert_{2}^{4}
    \right)
    = 9.81 \cdot
      \left(\frac{8}{3} + 32 - \frac{64}{3} - 8 + \frac{8}{3} \right)
    = 9.81 \cdot 8 \approx 78
\end{align*}

\textbf{Ответ:} \SI{78}{\kilo\newton}

\subsection{Задача № 2 В}

Найдите координаты центра масс плоской однородной фигуры,
ограниченной дугой синусоиды \(y = \sin x\)
и отрезком оси \(Ox\) (\(0 \le x \le \pi\)).

\subsubsection{Решение}

\begin{figure}[htbp]
  \centering
  \asyinclude{asy/2-c-1.asy}
  \caption{Графическая иллюстрация}\label{fig:2c}
\end{figure}

\paragraph{Математическая модель}
Центр масс вычисляется следующим образом:
\[
  \vec{r}_{c} =
    {\left(\int_{D} \rho(\vec{r}) \diff S\right)}^{-1}
    \int_{D} \rho(\vec{r}) \vec{r} \diff S
\]
где \(\rho(\vec{r})\) --- плотность.
В данной задаче плотность --- константа, поэтому справедливо
\begin{equation} \label{eq:2c-main}
  \vec{r}_{c} =
    {\left(\rho \int_{D} \diff S\right)}^{-1}
    \rho \int_{D} \vec{r} \diff S =
    {\left(\int_{D} \diff S\right)}^{-1}
    \int_{D} \vec{r} \diff S
\end{equation}

Так как вычисления производятся в ДПСК, то
\(\vec{r} = x \vec{i} + y \vec{j}\).
Пользуясь свойствами линейности интеграла, получаем:
\begin{equation} \label{eq:2c-main-1}
  \int_{D} \vec{r} \diff S
    = \int_{D} x \vec{i} + y \vec{j} \diff S
    = \int_{D} x \vec{i} \diff S + \int_{D} y \vec{j} \diff S
    = \vec{i} \int_{D} x \diff S + \vec{j} \int_{D} y \diff S
\end{equation}

Для каждого из полученных интегралов справедливо:
\begin{align}
  \int_{D} x \diff S = \int_{0}^{\pi} \diff x \int_{0}^{\sin x} x \diff y
  \label{eq:2c-1} \\
  \int_{D} y \diff S = \int_{0}^{\pi} \diff x \int_{0}^{\sin x} y \diff y
  \label{eq:2c-2} \\
  \int_{D} \diff S = \int_{0}^{\pi} \diff x \int_{0}^{\sin x} \diff y
  \label{eq:2c-3}
\end{align}

\paragraph{Вычисления}

Для начала вычислим интеграл по формуле \ref{eq:2c-3}:
\begin{equation} \label{eq:2c-3-calc}
\begin{split}
  \int_{D} \diff S
    &= \int_{0}^{\pi} \diff x \int_{0}^{\sin x} \diff y
     = \int_{0}^{\pi} \sin x \diff x
     = - \cos x \bigg\rvert_{0}^{\pi}
     = - \cos \pi + \cos 0 = 2
\end{split}
\end{equation}

Вычислим интегралы по формулам \ref{eq:2c-1} и \ref{eq:2c-2}:
\begin{equation} \label{eq:2c-1-calc}
\begin{split}
  \int_{D} x \diff S
  &= \int_{0}^{\pi} \diff x \int_{0}^{\sin x} x \diff y
   = \int_{0}^{\pi} x \sin x \diff x
   =
    \begin{bmatrix}
      u = x & v = - \cos x \\
      \diff u = \diff x & \diff v = \sin x \diff x
    \end{bmatrix} \\
  &= \bigg( -x \cos x \bigg) \bigg\rvert_{0}^{\pi}
   + \int_{0}^{\pi} \cos x \diff x
   = - \pi \cos\pi + \bigg( \sin x \bigg) \bigg\rvert_{0}^{\pi}
   = \pi
\end{split}
\end{equation}


\begin{equation} \label{eq:2c-2-calc}
\begin{split}
  \int_{D} y \diff S
  &= \int_{0}^{\pi} \diff x \int_{0}^{\sin x} y \diff y
   = \frac{1}{2} \int_{0}^{\pi} \sin^{2}(x) \diff x
   = \frac{1}{4} \int_{0}^{\pi} (1 - \cos(2x) \diff x \\
  &= \frac{1}{4}
     \left.\left(x - \frac{1}{2} \sin(2x) \right)\right\rvert_{0}^{\pi}
   = \frac{1}{4} \pi
\end{split}
\end{equation}

Подставим полученные значения в формулы \ref{eq:2c-main} и \ref{eq:2c-main-1}:
\begin{equation*}
\begin{split}
  \vec{r}_{c}
    &= {\left(\int_{D} \diff S\right)}^{-1}
      \int_{D} \vec{r} \diff S \\
    &= {\left(\int_{D} \diff S\right)}^{-1}
      \left(\vec{i} \int_{D} x \diff S + \vec{j} \int_{D} y \diff S\right)
    \xlongequal{
      \text{из \ref{eq:2c-1-calc}, \ref{eq:2c-2-calc} и \ref{eq:2c-3-calc}}
    }
     {2}^{-1} \left( \vec{i} \pi + \vec{j} \frac{\pi}{4} \right) \\
    &= \frac{\pi}{2} \vec{i} + \frac{\pi}{8} \vec{j}
\end{split}
\end{equation*}

\textbf{Ответ:} \(\left(\frac{\pi}{2}; \frac{\pi}{8}\right)\).


%
%\section{Задача № 3}
%
%Найти площадь фигуры, образованной графиком функции \(f(x)\) и её асимптотами,
%если
%\[
%  f(x) = \frac{x^4 - 4 x^3 + 6 x^2 - 4 x + 16}{3 {(x-1)}^3}.
%\]
%
%Проведите исследование:
%\begin{enumerate}
%\item Найдите асимптоты.
%\item Изобразите фигуры на рисунке (эскиз).
%\item Вычислите площадь фигуры.
%\item Запишите ответ.
%\end{enumerate}

\end{document}
